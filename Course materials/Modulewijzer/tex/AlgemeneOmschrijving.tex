\section{Algemene omschrijving}
	\subsection{Inleiding}
		In de logica is een afleidingsregel een regel die uit een aantal proposities een propositie afleidt. De proposities waar de propositie uit afgeleid wordt, worden de premissen genoemd en de afgeleide propositie de conclusie: de conclusie wordt geconcludeerd (of afgeleid) uit de premissen.
	
		Afleidingsregels zijn van toepassing aan een lange lijst van onderwerpen: kunstmatige intelligentie, ontleding, wiskundige bewijzen, type systemen, semantiek van programmeertalen, en zo voort.
		
		Afleidingsregels zijn dus een belangrijke middel om de regels van bewerking van een complexe systeem op een zeer abstracte niveau formeel te kunnen weergeven. \\

	\subsection{Relatie met andere onderwijseenheden}
		Op deze module wordt voortgebouwd in de module TINWIS08. \\

	\subsection{Leermiddelen}
		Verplicht:
		\begin{itemize}
			\item Presentaties die gebruikt worden in de hoorcolleges (pdf): te vinden op N@tschool
		\end{itemize}
		Facultatief:
		\begin{itemize}
			\item Boek: Boolos, George; Burgess, John; Jeffrey, Richard C. (2007). Computability and logic. Cambridge: Cambridge University Press
			\item John C. Reynolds (2009) [1998]. Theories of Programming Languages. Cambridge University Press. \\
			\item Text editors: Emacs, Notepad++, etc.
		\end{itemize}
		