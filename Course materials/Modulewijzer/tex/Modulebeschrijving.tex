\section*{Modulebeschrijving}
\begin{tabularx}{\textwidth}{|>{\columncolor{lichtGrijs}} p{.26\textwidth}|X|}
	\hline
	\textbf{Modulenaam:} & \modulenaam\\
	\hline
	\textbf{Modulecode: }& \modulecode\\
	\hline
	\textbf{Aantal studiepunten \newline en studiebelastinguren:} & Deze module levert \stdPunten studiepunten op, hetgeen overeenkomt met 56 uur.
	\begin{itemize}
		\item 8 $\times$ 120 minuten hoorcollege
		\item 2 $\times$ 120 minuten eindtoets
		\item het rest is zelfstudie
	\end{itemize} \\
	\hline
	\textbf{Vereiste voorkennis:}&Alle modules wiskunde uit het eerste jaar en de eerste helft van het tweede jaar.\\
	\hline
	\textbf{Werkvorm:} & Hoorcollege en practicum \\
	\hline
	\textbf{Toetsing:} & Schriftelijk toets \\
	\hline
	\textbf{Leermiddelen:} & Boolos, George; Burgess, John; Jeffrey, Richard C. (2007). Computability and logic. Cambridge: Cambridge University Press \\
John C. Reynolds (2009) [1998]. Theories of Programming Languages. Cambridge University Press. \\
	\hline
	\textbf{Leerdoelen:}&
	\begin{itemize}
		\item Het kunnen lezen (L), bewerken (B), en realiseren (R) van deductie systemen.
	\end{itemize} \\
	\hline
	\textbf{Inhoud:}&
	\begin{itemize}
		\item Kort geschiedkundig overzicht van logica in het gebied van programmeren;
		\item Deductie systemen;
		\item Toepassingen van deductie systemen.
	\end{itemize}\\
	\hline
	\textbf{Opmerkingen:}&\\
	\hline
	\textbf{Modulebeheerder:} & \author\\
	\hline
	\textbf{Datum:} & \today \\
	\hline
\end{tabularx}
\newpage
